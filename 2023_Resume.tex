%%%%%%%%%%%%%%%%%%%%%%%%%%%%%%%%%%%%%%%%%
% Umer936 Resume
% XeLaTeX
%
% Original author:
% Adrien Friggeri (adrien@friggeri.net)
% https://github.com/afriggeri/
%
% CV of:
% Umer Salman (umer936@gmail.com)
% https://github.com/umer936/My-CV
%
% License:
% CC BY-NC-SA 3.0 (http://creativecommons.org/licenses/by-nc-sa/3.0/)
%
%%%%%%%%%%%%%%%%%%%%%%%%%%%%%%%%%%%%%%%%%

\documentclass[]{friggeri-cv} % Add 'print' as an option into the square bracket to remove colors for printing
\usepackage[super]{nth}
\usepackage{lastpage}
\usepackage{fancyhdr}
\usepackage{datetime}
\usepackage{units}
\long\def\/*#1*/{}
\pagestyle{fancy}
\fancyhf{} % to clear existing header/footer if you don't want it
\rfoot{\monthname[\month] \the\year}
\renewcommand\headrulewidth{0pt}

\begin{document}
\header{umer}{salman}{\noindent\rule{10cm}{0.4pt}} % Your name and current job title/field
% \rfoot{\thepage\ of \pageref{LastPage}}

%----------------------------------------------------------------------------------------
%	SIDEBAR SECTION
%----------------------------------------------------------------------------------------

\begin{aside} % In the aside, each new line forces a line break
	\section{about}
	San Antonio, TX 78257
	USA
	~
	\href{tel:(936) 463-8626}{(936) 463 8626}
	~
	\href{mailto:umer936@gmail.com}{umer936@gmail.com}
	\href{mailto:umer936@utexas.edu}{umer936@utexas.edu}
	\href{http://umer936.com}{umer936.com}
	~
	\href{http://github.com/umer936}{github.com/umer936}
	\href{http://facebook.com/Umer936}{fb://Umer936}
	~
	\section{programming}
	CSS \& HTML
	{\color{red} $\varheartsuit$} PHP (Laravel, CakePHP)
	MySQL
	Bootstrap
	JavaScript, jQuery
	Java, C, C++
	\LaTeX \ (this r\'esum\'e)
	Python
	Assembly
	MATLAB
	LabVIEW
	Git, SVN, Mercurial
	~
	\section{operating systems}
	Linux
	Android
	OS X
	iOS
	Windows
	~
	\section{software skills}
	SolidWorks
	Burp Suite
	ROS2
	Gazebo
	OpenCV
	ArduPilot
	ADB/Fastboot
\end{aside}

%----------------------------------------------------------------------------------------
%	EDUCATION SECTION
%----------------------------------------------------------------------------------------

\section{education}
\vspace{-12pt}

\begin{entrylist}

	\entry
	{2016--2020}
	{BS in Electrical and Computer Engineering - Fall 2020}
	{The University of Texas at Austin}
	{
		\emph{Tech Cores: Software Engineering/Design and Academic Enrichment + 
		Minor: Business}
		\begin{itemize}
			\item \textbf{Embedded Systems Lab:} Created a USB 10-Keyless RGB Keyboard with Macro Recording using TM4C microcontroller \href{https://youtu.be/fSX6dDx9jvY}{\underline{[https://youtu.be/fSX6dDx9jvY]}}
			\item \textbf{Entrepreneurship Senior Design (CTO):} Built a wearable + software platform to collect and process physiological
			indicators to assist in diagnosing mental health issues in children \href{https://youtu.be/lP6gZINqRAU}{\underline{[Video]}}
			\begin{itemize}
				\item Designed PCB integrating an Adafruit Feather with a SD-card, Bluetooth module, galvanic skin response sensor, heart rate sensor, microphone, and battery
				\item Built Azure/Tableau web platform for healthcare professionals to view data
				\item Formed a business plan and presented to biomedical industry experts \href{https://youtu.be/n4y6X7Hgetc}{\underline{[Video]}}
			\end{itemize}
		\end{itemize}
	}
	
\end{entrylist}


\vspace{-5pt}
\vspace{-5pt}
\section{experience}
\vspace{-10pt}

\begin{entrylist}

\entry
{2021--Now}
{Southwest Research Institute}
{San Antonio, Texas}
{\emph{Research Computer Scientist}
	\begin{itemize}
		\item Develop Science Operations and Data Analysis Systems in Space Science Division (Div 15)
		\begin{itemize}
			\item Build modular full-stack interactive web apps for 50+ spacecraft and instruments including LAMP, Lucy, and Europa Clipper. Ensure swift deployment for new missions with smaller budgets and timelines, such as CubeSats, without compromising functionality
			\item Engineer mission-critical support software for flyby planning, data processing \& visualization, access control management, and legacy/modern software (IDL, FORTRAN, C++, Python) integration. Creates personalized mission data dashboards, providing Health and Safety plots for engineers and science data for the scientists
		\end{itemize}
		\item Researched and developed containerization (Docker/Kubernetes) of space science tools through an Internal R\&D grant, allowing for intelligent batch processing on HPCs or AWS as needed
		\begin{itemize}
			\item Presented insights and implementations at CakeFest 2023 Conference \href{https://www.youtube.com/live/4KB92R7UQc8?si=sA3sVMHHHfgKZKgp&t=19699}{\underline{[Video]}}, \href{https://umer936.com/cakefest-2023}{\underline{[Slides]}} and Data, Analysis, and Software in Heliophysics (DASH) \href{https://zenodo.org/doi/10.5281/zenodo.8412468}{\underline{[DOI: 10.5281/zenodo.8412469]}}
		\end{itemize}
		\item Created React UIs for a EW radar mission in Defense and Intelligence Solutions Division (Div 16)
	\end{itemize}
}

\vspace{-5pt}

\entry
{2019, 2020}
{National Geospacial-Intelligence Agency}
{St. Louis, Missouri; San Antonio, Texas (WFH)}
{\emph{Cybersecurity/Software Development Intern - Clearance: TS/SCI}
	\begin{itemize}
		\item Created automated rulesets and scripts for Detect and Incident Response teams
		\item Developed a NodeJS grammar tool for classified environments to correct ~6k writing mistakes and improve editing efficiency across NGA's Weekly Activity Reports, reports to Congress, etc
		\item Assisted in cyber-deception (honeypot ``network devices'') tool analysis
	\end{itemize}
}
\vspace{-5pt}
	
	%-----------------------------------------------
	
	\entry
	{2018}
	{Visa Inc.}
	{Austin, Texas}
	{\emph{Security Engineering Intern} \\
			Created automated penetration testing suite using Burp Suite and Python to increase security while decreasing time in security testing stage
			\begin{itemize}
				\item Tool tests Visa products \& APIs for vulnerabilities such as XSS and Clickjacking
				\item Learns from cybersecurity team to eliminate false positives
			\end{itemize}
	}

\end{entrylist}

\vspace{-5pt}
\section{extracurriculars}
\vspace{-10pt}

\begin{entrylist}
	
	\entry
	{2017--2020}
	{Texas Aerial Robotics (TAR)}
	{The University of Texas at Austin}
	{
		\emph{Founder and President}
		\begin{itemize}
			\item Led a 40 person org. to compete in the International Aerial Robotics Competition
			\item Build and program cutting-edge, fully autonomous quadcopters
			\item Use computer vision, LiDAR, and optical flow to target and interact with moving ground robots in a GPS-denied environment
			\item Research drone swarming and drone control through human voice and gestures
			\item Develop abilities to interact with modules on moving reference frames (boats) 2mi away
		\end{itemize}
	}
	
\end{entrylist}

\vspace{-25pt}

\end{document}
